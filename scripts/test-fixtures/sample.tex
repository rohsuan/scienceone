\documentclass{book}
\usepackage{amsmath}
\usepackage{amssymb}

\title{Sample STEM Manuscript}
\author{Test Author}

\begin{document}

\maketitle

\chapter{Introduction}

This chapter introduces fundamental concepts in physics and mathematics.
Einstein's famous mass-energy equivalence is given by the inline expression
$E = mc^2$, which unifies energy and mass through the speed of light.

A classic result in analysis is the Gaussian integral:
\begin{equation}
\int_0^\infty e^{-x^2} \, dx = \frac{\sqrt{\pi}}{2}
\end{equation}

This integral arises naturally in probability theory and Fourier analysis.

\chapter{Linear Algebra}

Linear algebra provides the language of modern machine learning and physics.
A vector $\vec{v} \in \mathbb{R}^n$ can be represented as a column matrix.

Consider the rotation matrix in two dimensions:
\begin{equation}
R(\theta) = \begin{pmatrix} \cos\theta & -\sin\theta \\ \sin\theta & \cos\theta \end{pmatrix}
\end{equation}

The eigenvalues of $R(\theta)$ are $e^{\pm i\theta}$, reflecting the complex
structure underlying real-valued rotations. For any $\vec{v} \in \mathbb{R}^n$,
norms satisfy $\|\vec{v}\| \geq 0$.

\chapter{Calculus}

The Riemann zeta function evaluated at $s = 2$ gives the famous Basel problem result:
\begin{equation}
\sum_{n=1}^{\infty} \frac{1}{n^2} = \frac{\pi^2}{6}
\end{equation}

The derivative of the sine function satisfies $\frac{d}{dx} \sin(x) = \cos(x)$,
a result that can be proved from the limit definition. More generally,
the fundamental theorem of calculus connects differentiation and integration.

\end{document}
